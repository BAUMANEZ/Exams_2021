\makeatletter
\def\input@path{{../Template/}{  }} % Empty brackets for images path
\makeatother

\documentclass[12pt, a4paper]{article}

\usepackage[utf8]{inputenc}
\usepackage[T1]{fontenc}
\usepackage[russian]{babel}
\usepackage[]{float}

\usepackage[oglav, boldsect, eqwhole, figwhole,remarks, hyperref, hyperprint]{fn2kursstyle}

%начало
\begin{document}
    \tableofcontents 
    
    \NumTask{Параметризованные кривые в пространстве. Регулярные и особые точки параметризованных кривых. Гладкие кривые }

    \Picture{11.jpeg}{1}
    \Picture{12.jpeg}{1}
    \Picture{13.jpeg}{1}
    \Picture{14.jpeg}{1}
    \Picture{15.jpeg}{1}

    \NumTask{Репараметризация кривых. Натуральный парамет кривой. Существование натуральной параметризации на гладкой регулярной кривой. Свойства векторов скорости и ускорения кривой, отнесенной к натуральному параметру}

    \Picture{21.jpeg}{1}
    \Picture{22.jpeg}{1}
    \Picture{23.jpeg}{1}

    \NumTask{Кривизна кривой. Радиус кривизны, центр кривизны, вектор кривизны}
   
    \Picture{31.jpeg}{1}
    \Picture{32.jpeg}{0.8}

    \NumTask{Репер Френе}

    \Picture{41.jpeg}{0.9}
    \Picture{42.jpeg}{0.9}

    \NumTask{Сопровождающий трехгранник кривой}

    \Picture{51.jpeg}{1}
    \Picture{52.jpeg}{1}

    \NumTask{Формулы Френе. Кручение кривой}

    \Picture{61.jpeg}{1}
    \Picture{62.jpeg}{1}

    \pagebreak

    \NumTask{Геометрический смысл кривизны и кручения (4 теоремы)}

    \Picture{71.jpeg}{0.9}
    \Picture{72.jpeg}{0.9}
    \Picture{73.jpeg}{1}
    \Picture{74.jpeg}{1}

    \pagebreak

    \NumTask{Механический смысл формул Френе. Вектор Дарбу}

    \Picture{81.jpeg}{1}
    \Picture{82.jpeg}{1}
    \Picture{83.jpeg}{1}

    \pagebreak

    \NumTask{Кривизна и кручение кривой, отнесенной к произвольному параметру.}

    \Picture{91.jpeg}{1}
    \Picture{92.jpeg}{1}
    \Picture{93.jpeg}{1}

    \pagebreak

    \NumTask{Формулы для вычисления кривизны плоских кривых}

    \Picture{101.jpeg}{1}
    \Picture{102.jpeg}{1}

    \pagebreak

    \NumTask{Репер Френе кривой, отнесенной к произвольному параметру}

    \Picture{111.jpeg}{0.9}
    \Picture{112.jpeg}{1}

    \NumTask{Натуральные уравения кривой. Теорема существования и единственности кривой с данными кривизной и кручением (доказать единственность)}

    \Picture{121.jpeg}{1}
    \Picture{122.jpeg}{1}

    \NumTask{Параметризованные поверхности в пространстве. Регулярные и особые точки параметризованныз поверхностей. Гладкие поверхности}

    \Picture{131.png}{1}
    \Picture{132.png}{1}
    \Picture{133.png}{1}
    \Picture{134.png}{1}

    \pagebreak
   
    \NumTask{Криволинейные координаты и координатная сеть на поверхности. Касательное пространство}

    \Picture{141.png}{1}
    \Picture{142.png}{1}
    \Picture{143.png}{1}
    \Picture{144.png}{1}


    \NumTask{Замена криволинейной системы координат на поверхности. Преобразование координат касательного вектора при замене криволинейно системы координат на поверхности}

    \Picture{151.png}{1}
    \Picture{152.png}{1}
    \Picture{153.png}{1}
    \Picture{154.png}{1}
    \Picture{155.png}{1}

    \addtocontents{toc}{\protect\newpage}

    \pagebreak

    \NumTask{Задача о вычислении длины кривой на поверхности. Первая квадратичная форма поверхности. Положительная определенность первой квадратичной формы. Примеры: первая квадратичная форма графика функции и поверхности вращения. Закон преобразования первой квадратичной формы при замене криволинейной системы координат на поверхности}

    \Picture{161.png}{1}
    \Picture{162.png}{1}
    \Picture{163.png}{1}
    \Picture{164.png}{1}
    \Picture{165.png}{1}
    \Picture{166.png}{1}
    \Picture{167.png}{1}
    \Picture{168.png}{1}

    \pagebreak

    \NumTask{Угол между кривыми на поверхности}

    \Picture{171.png}{0.9}
    \Picture{172.png}{0.9}


    \NumTask{Внутренняя геометрия поверхности. Изометрии. Изометричные поверхности. Теорема о квадратичных формах изометричных поверхностей}

    \Picture{181.png}{1}

    \NumTask{Вторая квадратичная форма поверхности. Примеры: вторая квадратичная форма графика функции и поверхности вращения. Закон преобразования второй квадратичной формы при замене КСК на поверхности}

    \Picture{191.png}{1}
    \Picture{192.png}{0.9}
    \Picture{193.png}{0.9}
    \Picture{194.png}{1}
    \Picture{195.png}{0.9}
    \Picture{196.png}{0.9}

    \NumTask{Геометрический смысл второй квадратичной формы поверхности}

    \Picture{201.png}{0.8}
    \Picture{202.png}{0.8}


    \NumTask{Классификация точек поверхности (эллиптические, гиперболические, параболические и точки уплощения). Расположение поверхности в окрестности эллиптической или гиперболической точки относительно касательной плоскости в этой точке}

    \Picture{211.png}{1}
    \Picture{212.png}{1}

    \pagebreak

    \NumTask{Основная формула для кривизны кривой на поверхности}

    \Picture{221.png}{1}
    \Picture{222.png}{1}

    \pagebreak
    
    \NumTask{Нормальная кривизна поверхности}

    \Picture{231.png}{0.7}
    \Picture{232.png}{0.9}


    \NumTask{Главные направления и главные кривизны поверхности: определение, способ нахождения, свойства}

    \Picture{241.png}{1}
    \Picture{242.png}{1}
    \Picture{243.png}{1}
    \Picture{244.png}{1}

    \pagebreak

    \NumTask{Формула Эйлера}

    \Picture{251.png}{1}

    \NumTask{Гауссова и средняя кривизны поверхности. Связь средней кривизны и нормальной кривизны. Формулы, выражающие гауссову и среднюю кривизны через коэффициенты первой и второй квадратичных форм поверхности. Классификация точек поверхности по знаку гауссовой кривизны}

    \Picture{261.png}{1}
    \Picture{262.png}{1}
    \Picture{263.png}{1}
    \Picture{264.png}{1}

    \pagebreak

    \NumTask{Минимальные поверхности. Теорема о средней кривизне минимальных поверхностей. Вывод уравнения минимальных поверхностей}

    \Picture{271.png}{1}
    \Picture{272.png}{1}

    \pagebreak

    \addtocontents{toc}{\protect\newpage}

    \NumTask{Нормальная и геодезическая кривизны кривой на поверхности. Формулы для вычисления геодезической кривизны}

    \Picture{281.png}{0.9}
    \Picture{282.png}{0.8}
    \Picture{283.png}{1}

    \pagebreak

    \NumTask{Геодезические линии на поверхности. Теорема о главной нормали геодезической. Уравнения геодезических. Теорема существования и единственности. Экстремальное свойство геодезических. Примеры геодезических}

    \Picture{291.png}{1}
    \Picture{292.png}{1}
    \Picture{293.png}{1}
    \Picture{294.png}{1}
    \Picture{295.png}{1}

    \addtocontents{toc}{\protect\newpage}
    
 \end{document}