\makeatletter
\def\input@path{{../Template/}{  }} % Empty brackets for images path
\makeatother

\documentclass[12pt, a4paper]{article}

\usepackage[utf8]{inputenc}
\usepackage[T1]{fontenc}
\usepackage[russian]{babel}
\usepackage[]{float}

\usepackage[oglav, boldsect, eqwhole, figwhole,remarks, hyperref, hyperprint]{fn2kursstyle}

%начало
\begin{document}
    \tableofcontents 

    \pagebreak
    
    \NumTask{Параметризованные кривые в пространстве. Регулярные и особые точки параметризованных кривых. Гладкие кривые }

    \Picture{11.jpeg}{1}
    \Picture{12.jpeg}{1}
    \Picture{13.jpeg}{1}
    \Picture{14.jpeg}{1}
    \Picture{15.jpeg}{1}

    \NumTask{Репараметризация кривых. Натуральный парамет кривой. Существование натуральной параметризации на гладкой регулярной кривой. Свойства векторов скорости и ускорения кривой, отнесенной к натуральному параметру}

    \Picture{21.jpeg}{1}
    \Picture{22.jpeg}{1}
    \Picture{23.jpeg}{1}

    \NumTask{Кривизна кривой. Радиус кривизны, центр кривизны, вектор кривизны}
   
    \Picture{31.jpeg}{1}
    \Picture{32.jpeg}{0.8}

    \pagebreak

    \NumTask{Репер Френе}

    \Picture{41.jpeg}{0.9}
    \Picture{42.jpeg}{0.9}

    \NumTask{Сопровождающий трехгранник кривой}

    \Picture{51.jpeg}{1}
    \Picture{52.jpeg}{1}

    \NumTask{Формулы Френе. Кручение кривой}

    \Picture{61.jpeg}{1}
    \Picture{62.jpeg}{1}

    \pagebreak

    \NumTask{Геометрический смысл кривизны и кручения (4 теоремы)}

    \Picture{71.jpeg}{0.9}
    \Picture{72.jpeg}{0.9}
    \Picture{73.jpeg}{1}
    \Picture{74.jpeg}{1}

    \pagebreak

    \NumTask{Механический смысл формул Френе. Вектор Дарбу}

    \Picture{81.jpeg}{1}
    \Picture{82.jpeg}{1}
    \Picture{83.jpeg}{1}

    \pagebreak

    \NumTask{Кривизна и кручение кривой, отнесенной к произвольному параметру.}

    \Picture{91.jpeg}{1}
    \Picture{92.jpeg}{1}
    \Picture{93.jpeg}{1}

    \pagebreak

    \NumTask{Формулы для вычисления кривизны плоских кривых}

    \Picture{101.jpeg}{1}
    \Picture{102.jpeg}{1}

    \pagebreak

    \NumTask{Репер Френе кривой, отнесенной к произвольному параметру}

    \Picture{111.jpeg}{0.9}
    \Picture{112.jpeg}{1}

    \NumTask{Натуральные уравения кривой. Теорема существования и единственности кривой с данными кривизной и кручением (доказать единственность)}

    \Picture{121.jpeg}{1}
    \Picture{122.jpeg}{1}

    \NumTask{Параметризованные поверхности в пространстве. Регулярные и особые точки параметризованныз поверхностей. Гладкие поверхности}

    \Picture{131.png}{1}
    \Picture{132.png}{1}
    \Picture{133.png}{1}
    \Picture{134.png}{1}

    \pagebreak
   
    \NumTask{Криволинейные координаты и координатная сеть на поверхности. Касательное пространство}

    \Picture{141.png}{1}
    \Picture{142.png}{1}
    \Picture{143.png}{1}
    \Picture{144.png}{1}


    \NumTask{Замена криволинейной системы координат на поверхности. Преобразование координат касательного вектора при замене криволинейно системы координат на поверхности}

    \Picture{151.png}{1}
    \Picture{152.png}{1}
    \Picture{153.png}{1}
    \Picture{154.png}{1}
    \Picture{155.png}{1}

    \addtocontents{toc}{\protect\newpage}

    \pagebreak

    \NumTask{Задача о вычислении длины кривой на поверхности. Первая квадратичная форма поверхности. Положительная определенность первой квадратичной формы. Примеры: первая квадратичная форма графика функции и поверхности вращения. Закон преобразования первой квадратичной формы при замене криволинейной системы координат на поверхности}

    \Picture{161.png}{1}
    \Picture{162.png}{1}
    \Picture{163.png}{1}
    \Picture{164.png}{1}
    \Picture{165.png}{1}
    \Picture{166.png}{1}
    \Picture{167.png}{1}
    \Picture{168.png}{1}

    \pagebreak

    \NumTask{Угол между кривыми на поверхности}

    \Picture{171.png}{0.9}
    \Picture{172.png}{0.9}


    \NumTask{Внутренняя геометрия поверхности. Изометрии. Изометричные поверхности. Теорема о квадратичных формах изометричных поверхностей}

    \Picture{181.png}{1}

    \NumTask{Вторая квадратичная форма поверхности. Примеры: вторая квадратичная форма графика функции и поверхности вращения. Закон преобразования второй квадратичной формы при замене КСК на поверхности}

    \Picture{191.png}{1}
    \Picture{192.png}{0.9}
    \Picture{193.png}{0.9}
    \Picture{194.png}{1}
    \Picture{195.png}{0.9}
    \Picture{196.png}{0.9}

    \NumTask{Геометрический смысл второй квадратичной формы поверхности}

    \Picture{201.png}{0.8}
    \Picture{202.png}{0.8}


    \NumTask{Классификация точек поверхности (эллиптические, гиперболические, параболические и точки уплощения). Расположение поверхности в окрестности эллиптической или гиперболической точки относительно касательной плоскости в этой точке}

    \Picture{211.png}{1}
    \Picture{212.png}{1}

    \pagebreak

    \NumTask{Основная формула для кривизны кривой на поверхности}

    \Picture{221.png}{1}
    \Picture{222.png}{1}

    \pagebreak
    
    \NumTask{Нормальная кривизна поверхности}

    \Picture{231.png}{0.7}
    \Picture{232.png}{0.9}


    \NumTask{Главные направления и главные кривизны поверхности: определение, способ нахождения, свойства}

    \Picture{241.png}{1}
    \Picture{242.png}{1}
    \Picture{243.png}{1}
    \Picture{244.png}{1}

    \pagebreak

    \NumTask{Формула Эйлера}

    \Picture{251.png}{1}

    \NumTask{Гауссова и средняя кривизны поверхности. Связь средней кривизны и нормальной кривизны. Формулы, выражающие гауссову и среднюю кривизны через коэффициенты первой и второй квадратичных форм поверхности. Классификация точек поверхности по знаку гауссовой кривизны}

    \Picture{261.png}{1}
    \Picture{262.png}{1}
    \Picture{263.png}{1}
    \Picture{264.png}{1}

    \pagebreak

    \NumTask{Минимальные поверхности. Теорема о средней кривизне минимальных поверхностей. Вывод уравнения минимальных поверхностей}

    \Picture{271.png}{1}
    \Picture{272.png}{1}

    \pagebreak

    \addtocontents{toc}{\protect\newpage}

    \NumTask{Нормальная и геодезическая кривизны кривой на поверхности. Формулы для вычисления геодезической кривизны}

    \Picture{281.png}{0.9}
    \Picture{282.png}{0.8}
    \Picture{283.png}{1}

    \pagebreak

    \NumTask{Геодезические линии на поверхности. Теорема о главной нормали геодезической. Уравнения геодезических. Теорема существования и единственности. Экстремальное свойство геодезических. Примеры геодезических}

    \Picture{291.png}{1}
    \Picture{292.png}{1}
    \Picture{293.png}{1}
    \Picture{294.png}{1}
    \Picture{295.png}{1}

    \NumTask{Линии кривизны. Уравнение линий кривизны. Асимптотические линии. Уравнение асимптотических линий. Примеры}

    \Picture{301.png}{1}
    \Picture{302.png}{1}
    \Picture{303.png}{1}
    \Picture{304.png}{1}
    \Picture{305.png}{1}
    \Picture{306.png}{1}

    \NumTask{Соприкасающийся параболоид. Вид соприкасающегося параболоида для точек разного типа}

    \Picture{311.png}{1}
    \Picture{312.png}{1}

    \NumTask{Основные уравнения теории поверхностей. Деривационные формулы. Теорема Бонне}

    \Picture{321.png}{1}
    \Picture{322.png}{1}
    \Picture{323.png}{1}

    \NumTask{Криволинейные системы координат в области евклидова пространства. Лемма о локальной системе координат. Замена криволинейной системы координат. Локальный базис криволинейной
    системы координат.}

    \Picture{331.png}{1}
    \Picture{332.png}{1}


    \NumTask{Касательное пространство, его базис. Преобразование координат касательного вектора,
    при замене криволинейной системы координат.}

    \Picture{341.png}{1}
    \Picture{342.png}{1}


    \NumTask{Длина кривой в криволинейной системе координат. Функции $ g_{ij} $‚ их геометрический смысл и свойства. Коэффициенты Ламе ортогональной системы координат. Вычисление углов и объемов в криволинейной системе координат.}

    \Picture{351.png}{1}
    \Picture{352.png}{1}
    \Picture{353.png}{1}
    \Picture{354.png}{1}

    \pagebreak


    \NumTask{Риманова метрика (метрический тензор) в области евклидова пространства. Римановы пространства. Длина кривой в римановом пространстве.}

    \Picture{361.png}{1}
    \Picture{362.png}{1}

    \pagebreak


    \NumTask{Скалярное произведение в касательном пространстве, порождаемое римановой метрикой.
    Угол между кривыми в римановом пространстве.}

    \Picture{371.png}{0.9}
    \Picture{372.png}{0.9}

    \pagebreak


    \NumTask{Объем области в римановом пространстве.}

    \Picture{381.png}{1}
    \Picture{382.png}{1}

    \pagebreak


    \NumTask{Псевдоримановы (индефинитные) метрики. Псевдоримановы пространства.}

    \Picture{391.png}{1}
    
    \pagebreak


    \addtocontents{toc}{\protect\newpage}


    \NumTask{Гладкая k-мерная поверхность. Задача о вычислении длины кривой на поверхности. Инлуцированная метрика на поверхности.}

    \Picture{401.png}{1}
    \Picture{402.png}{1}

    \pagebreak

    \NumTask{Модель геометрии Лобачевского на, верхней полуплоскости.}

    \Picture{411.png}{1}
    
    \pagebreak


    \NumTask{ Координатное определение тензора и тензорного поля. Примеры тензоров и тензорных полей.}

    \Picture{421.png}{1}
    \Picture{422.png}{1}
    \Picture{423.png}{1}
    \Picture{424.png}{1}

    \pagebreak


    \NumTask{Задание тензора, (тензорного поля) его компонентами в некоторой системе координат.}

    \Picture{431.png}{1}
    \Picture{432.png}{1}

    \pagebreak

    \NumTask{Обратный тензорный признак.}

    \Picture{441.png}{0.9}
    \Picture{442.png}{1}

    \pagebreak

    \NumTask{Алгебраические операции над тензорами и тензорными полями, их свойства.}

    \Picture{451.png}{1}
    \Picture{452.png}{1}
    \Picture{453.png}{1}
    \Picture{454.png}{1}
    \Picture{455.png}{1}
    \Picture{456.png}{1}
    \Picture{457.png}{1}

    \pagebreak


    \NumTask{Симметричные и кососимметричные тензоры, их свойства.}

    \Picture{461.png}{1}
    \Picture{462.png}{1}
    \Picture{463.png}{1}

    \pagebreak


    \NumTask{ Тензоры как полилинейные функции. Модуль  $ {T_p}^q (L) $. Связь координатного и алгебраического определений тензоров.}

    \Picture{471.png}{1}

    \pagebreak


    \NumTask{Базис модуля $ {T_p}^q (L) $, координаты (компоненты) тензора.}

    \Picture{481.png}{1}
    \Picture{482.png}{1}

    \pagebreak

    \NumTask{Базис модуля тензорных полей в области евклидова пространства. Инвариантная форма записи тензорного поля.}

    \Picture{491.png}{0.9}
    \Picture{492.png}{0.9}

    \pagebreak


    \NumTask{Векторные поля. Производная по направлению векторного поля, ее свойства. Инвариантная форма записи векторного поля. Коммутатор векторных полей, его свойства. Алгебра Ли векторных полей.}

    \Picture{501.png}{0.8}
    \Picture{502.png}{1}
    \Picture{503.png}{0.9}


    \NumTask{Ковариантное (инвариантное) дифференцирование тензорных полей. Теорема, сушествования тензорной операции дифференцирования тензорных полей (вычисления провести для векторных, ковекторных и операторных полей).}

    \Picture{511.png}{0.75}
    \Picture{512.png}{1}
    \Picture{513.png}{1}
    \Picture{514.png}{1}

    \pagebreak

    \NumTask{Символы Кристоффеля, их кинематический смысл. Закон преобразования символов Кристоффеля.}

    \Picture{521.png}{1}
    \Picture{522.png}{1}

    \pagebreak


    \NumTask{Символы Кристоффеля в евклидовом пространстве $ \mathbb{R}^n $.}

    \Picture{531.png}{1}
    \Picture{532.png}{1}

    \pagebreak


    \NumTask{Аффинная связность. Риманова, связность. Символы Кристоффеля римановой связности.}

    \Picture{541.png}{1}
    \Picture{542.png}{1}

    \pagebreak


    \NumTask{Ковариантное дифференцирование влоль кривой. Параллельный перенос векторов.}

    \Picture{551.png}{1}
    \Picture{552.png}{1}

    \pagebreak


    \addtocontents{toc}{\protect\newpage}

    \NumTask{Геодезические. Уравнение геодезических.}

    \Picture{561.png}{0.9}
    \Picture{562.png}{0.9}

    \pagebreak


    \NumTask{ Внешние дифференциальные формы. Внешнее произведение дифференциальных форм. Внешний дифференциал, его свойства.}

    \Picture{571.png}{0.9}
    \Picture{572.png}{0.9}
    \Picture{573.png}{0.9}


    \NumTask{Градиент, ротор и дивергенция как внешние лифференциалы в комплексе де Рама, пространства $ \mathbb{R}^3 $}

    \Picture{581.png}{1}

    \pagebreak

    \NumTask{Дифференциальные операции векторного анализа в криволинейных координатах}

    \Picture{591.png}{1}
    \Picture{592.png}{1}
    \Picture{593.png}{1}
    
 \end{document}
