\makeatletter
\def\input@path{{../Template/}{  }} % Empty brackets for images path
\makeatother

\documentclass[12pt, a4paper]{article}

\usepackage[utf8]{inputenc}
\usepackage[T1]{fontenc}
\usepackage[russian]{babel}
\usepackage[]{float}

\usepackage[oglav, boldsect, eqwhole, figwhole,remarks, hyperref, hyperprint]{fn2kursstyle}

%начало
\begin{document}
    \tableofcontents 
    
    \NumTask{Параметризованные кривые в пространстве. Регулярные и особые точки параметризованных кривых. Гладкие кривые }

    \Picture{11.jpeg}{1}
    \Picture{12.jpeg}{1}
    \Picture{13.jpeg}{1}
    \Picture{14.jpeg}{1}
    \Picture{15.jpeg}{1}

    \NumTask{Репараметризация кривых. Натуральный парамет кривой. Существование натуральной параметризации на гладкой регулярной кривой. Свойства векторов скорости и ускорения кривой, отнесенной к натуральному параметру}

    \Picture{21.jpeg}{1}
    \Picture{22.jpeg}{1}
    \Picture{23.jpeg}{1}

    \NumTask{Кривизна кривой. Радиус кривизны, центр кривизны, вектор кривизны}
    
    \NumTask{Репер Френе}

    \NumTask{Сопровождающий трехгранник кривой}

    \NumTask{Формулы Френе. Кручение кривой}

    \NumTask{Геометрический смысл кривизны и кручения (4 теоремы)}

    \NumTask{Механический смысл формул Френе. Вектор Дарбу}

    \NumTask{Кривизна и кручение кривой, отнесенной к произвольному параметру.}

    \NumTask{Формулы для вычисления кривизны плоских кривых}

    \NumTask{Репер Френе кривой, отнесенной к произвольному параметру}

    \NumTask{Натуральные уравения кривой. Теорема существования и единственности кривой с данными кривизной и кручением (доказать единственность)}

    \NumTask{Параметризованные поверхности в пространстве. Регулярные и особые точки параметризованныз поверхностей. Гладкие поверхности}

    \NumTask{Криволинейные координаты и координатная сеть на поверхности. Касательное пространство}

    \NumTask{Замена криволинейной системы координат на поверхности. Преобразование координат касательного вектора при замене криволинейно системы координат на поверхности}

    \addtocontents{toc}{\protect\newpage}

    \NumTask{Задача о вычислении длины кривой на поверхности. Первая квадратичная форма поверхности. Положительная определенность первой квадратичной формы. Примеры: первая квадратичная форма графика функции и поверхности вращения. Закон преобразования первой квадратичной формы при замене криволинейной системы координат на поверхности}

    \NumTask{Угол между кривыми на поверхности}

    \NumTask{Внутренняя геометрия поверхности. Изометрии. Изометричные поверхности. Теорема о квадратичных формах изометричных поверхностей}

    \NumTask{Вторая квадратичная форма поверхности. Примеры: вторая квадратичная форма графика функции и поверхности вращения. Закон преобразования второй квадратичной формы при замене КСК на поверхности}

    \NumTask{Геометрический смысл второй квадратичной формы поверхности}

    \NumTask{Классификация точек поверхности (эллиптические, гиперболические, параболические и точки уплощения). Расположение поверхности в окрестности эллиптической или гиперболической точки относительно касательной плоскости в этой точке}

    \NumTask{Основная формула для кривизны кривой на поверхности}
    
    \NumTask{Нормальная кривизна поверхности}

    \NumTask{Главные направления и главные кривизны поверхности: определение, способ нахождения, свойства}

    \NumTask{Формула Эйлера}

    \NumTask{Гауссова и средняя кривизны поверхности. Связь средней кривизны и нормальной кривизны. Формулы, выражающие гауссову и среднюю кривизны через коэффициенты первой и второй квадратичных форм поверхности. Классификация точек поверхности по знаку гауссовой кривизны}

    \NumTask{Минимальные поверхности. Теорема о средней кривизне минимальных поверхностей. Вывод уравнения минимальных поверхностей}

    \addtocontents{toc}{\protect\newpage}

    \NumTask{Нормальная и геодезическая кривизны кривой на поверхности. Формулы для вычисления геодезической кривизны}

    \NumTask{Геодезические линии на поверхности. Теорема о главной нормали геодезической. Уравнения геодезических. Теорема существования и единственности. Экстремальное свойство геодезических. Примеры геодезических}

    \addtocontents{toc}{\protect\newpage}
    
 \end{document}