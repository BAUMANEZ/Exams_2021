\documentclass[12pt, a4paper]{article}

\usepackage[utf8]{inputenc}
\usepackage[T1]{fontenc}
\usepackage[russian]{babel}
\usepackage[]{float}

\usepackage[oglav, boldsect, eqwhole, figwhole, %
   remarks, hyperref, hyperprint]{fn2kursstyle}


\frenchspacing
\righthyphenmin=2

%начало
\begin{document}
\centerline{\bf Билеты к летней сессии 4-го семестра}
\centerline{\bf Теоретическая механика}
Рекомендации к использованию и некоторые пояснения: 

1. В данном файле собрана вся теоретическая часть к экзамену из учебника Колесникова~К.\,С.<<Курс теоретической механики>>. Все вопросы расписаны подробно, поэтому имеет смысл пробежаться по всему файлу и выделить в pdf самое важное. Если так оказалось, что в файле по вопросу материал подобран неверно, то это можно исправить в исходнике <<Термех.tex>> с помощью функций NumTask и Picture.

2. Некоторые вопросы не содержатся в учебнике, поэтому материал был найден в сторонних источниках. 

3. К некоторым вопросам я прикрепил вдобавок к учебнику конспекты лекций. Лекции более короткие и лаконичные. 

4. Оглавление стандартное. Нажав на вопрос, вас отправляет на страницу, где он находится. 

5. Некоторые изображения были умышленно сужены, так как они не помещались с предыдущими или следующими по размеру на одну страницу. Качество не должно было сильно испортиться. 

6. В среднем на один вопрос приходится 3 -- 4 изображения.

\tableofcontents 

\NumTask{Аксиомы динамики.}
\Picture{1.pdf}{0.9}
\newpage

\NumTask{Дифференциальные уравнения движения материальной точки в векторной форме и в проекциях на декартовы и естественные оси координат.}
\Picture{2.1.jpg}{1}
\Picture{2.2.jpg}{1}
\newpage

\NumTask{Две основные задачи динамики точки. Интегралы уравнений движений точки.}
\Picture{3.1.jpg}{1}
\Picture{3.2.jpg}{1}
\Picture{3.3.jpg}{1}
\newpage

\NumTask{Дифференциальные уравнения движения точки в неинерциальной системе отсчета (Динамика относительного движения).}
\Picture{4.1.jpg}{1}
\Picture{4.2.jpg}{1}
\newpage

\NumTask{Принцип относительности Галилея - Ньютона.}
\Picture{5.1.jpg}{1}

\NumTask{Центр масс механической системы. Теорема о движении центра масс.}
\Picture{6.1.jpg}{1}
\Picture{6.2.jpg}{1}
\Picture{6.3.jpg}{1}
\Picture{6.4.jpg}{1}
\newpage

\NumTask{Теорема об изменении количество движения в дифференциальной и интегральной формах.}
\Picture{7.1.jpg}{1}
\Picture{7.2.jpg}{1}
\Picture{7.3.jpg}{1}
\Picture{7.4.jpg}{0.7}
\Picture{7.5.jpg}{0.7}
\newpage

\NumTask{Законы сохранения количества движения механической системы.}
\Picture{8.1.jpg}{0.65}
\Picture{8.2.jpg}{0.7}
\newpage

\NumTask{Законы сохранения кинетического момента (момента количества движения) относительно центра и оси. Примеры.}
\Picture{9.1.jpg}{1}
\Picture{9.2.jpg}{1}
\Picture{9.3.jpg}{1}
\Picture{9.4.jpg}{1}
\newpage

\NumTask{Теоремы об изменении кинетического момента для точки и системы материальных точек.}
\Picture{10.1.jpg}{0.9}
\Picture{10.2.jpg}{1}
\Picture{10.3.jpg}{1}
\newpage

\NumTask{Кинетический момент твердого тела относительно оси вращения.}
\Picture{11.1.jpg}{0.6}
\Picture{11.2.jpg}{0.6}
\Picture{11.3.jpg}{1}

\NumTask{Теорема об изменении кинетического момента системы материальных точек в относительном движении по отношению к центру масс.}
\Picture{12.1.jpg}{0.7}
\Picture{12.2.jpg}{1}
\newpage

\NumTask{Кинетический момент твердого тела относительно неподвижной точки, его проекции на оси декартовой системы координат.}
\Picture{13.1.jpg}{0.8}
\Picture{13.2.jpg}{0.8}
\Picture{13.3.jpg}{1}
\Picture{13.4.jpg}{1}

\NumTask{Кинетический момент твердого тела относительно неподвижной точки в случае сферического движения.}
\Picture{14.1.jpg}{1}
\Picture{14.2.jpg}{1}
\Picture{14.3.jpg}{1}
\newpage

\NumTask{Вывод динамических уравнений Эйлера.}
\Picture{15.1.jpg}{1}
\Picture{15.2.jpg}{1}
\newpage

\NumTask{Вывод формулы для кинетического моменты системы материальных точек при сложном движении.}
\Picture{16.1.jpg}{0.6}
\Picture{16.2.jpg}{0.6}
\Picture{16.3.jpg}{1}
\Picture{16.4.jpg}{0.75}

\addtocontents{toc}{\protect\newpage}

\NumTask{Дифференциальное уравнение вращения твердого тела вокруг неподвижной оси.}
\Picture{17.1.jpg}{0.75}
\newpage

\NumTask{Дифференциальное уравнение плоского движения твердого тела.}
\Picture{18.1.jpg}{1}
\Picture{18.2.jpg}{1}
\Picture{18.3.jpg}{1}
\newpage

\NumTask{Элементарная и полная работа силы. Мощность. Работа равнодействующей силы.}
\Picture{19.1.jpg}{0.8}
\Picture{19.2.jpg}{0.8}
\Picture{19.3.jpg}{1}

\NumTask{Кинетическая энергия точки и системы материальных точек. Теорема Кёнинга.}
\Picture{20.1.jpg}{0.8}
\Picture{20.2.jpg}{0.8}
\Picture{20.3.jpg}{1}
\Picture{20.4.jpg}{1}
\Picture{20.5.jpg}{1}
\newpage

\NumTask{Кинетическиая энергия твердого тела с одной неподвижной точкой.}
\Picture{21.1.jpg}{1}
\Picture{21.2.jpg}{1}
\Picture{21.3.jpg}{1}
\newpage

\NumTask{Теоремы об изменении кинетической энергии точки и механической системы.}
\Picture{22.1.jpg}{1}
\Picture{22.2.jpg}{1}
\Picture{22.3.jpg}{1}
\newpage

\NumTask{Принцип Даламбера для точки и системы точек.}
\Picture{23.1.jpg}{1}
\Picture{23.2.jpg}{1}
\Picture{23.3.jpg}{1}
\Picture{23.4.jpg}{1}
\newpage

\NumTask{Закон сохранения полной механической энергии.}
\Picture{24.1.jpg}{1}
\Picture{24.2.jpg}{0.75}

\NumTask{Главный вектор и главный момент сил инерции в общем и частных случаях движения твердого тела.}
\Picture{25.1.jpg}{0.75}
\Picture{25.2.jpg}{1}
\Picture{25.3.jpg}{1}
\newpage

\NumTask{Связи и их классификация.}
\Picture{26.1.jpg}{0.8}
\Picture{26.2.jpg}{0.8}
\Picture{26.3.jpg}{0.85}
\Picture{26.4.jpg}{0.85}
\Picture{26.5.jpg}{0.85}
\newpage

\NumTask{Элементарная работа силы на возможном перемещении.}
\Picture{27.1.jpg}{0.9}
\Picture{27.2.jpg}{0.9}
\newpage

\NumTask{Возможные перемещения точки и механической системы. Принцип возможных перемещений.}
\Picture{28.1.jpg}{1}
\Picture{28.2.jpg}{1}
\Picture{28.3.jpg}{0.85}
\Picture{28.4.jpg}{0.85}
\newpage

\NumTask{Общее уравнение динамики.}
\Picture{29.1.jpg}{1}
\newpage

\NumTask{Обощенные силы механической системы и способы их вычисления.}
\Picture{30.1.jpg}{1}
\Picture{30.2.jpg}{0.9}
\Picture{30.3.jpg}{0.9}
\Picture{30.4.jpg}{0.9}
\Picture{30.5.jpg}{0.9}
\Picture{30.6.jpg}{0.8}
\newpage

\NumTask{Обощенные силы механической системы. Условия равновесия механической системы в обобщенных силах.}
\Picture{31.1.jpg}{0.8}
\Picture{31.2.jpg}{0.9}

\NumTask{Вывод уравнения Лагранжа второго рода.}
\Picture{32.1.jpg}{1}
\Picture{32.2.jpg}{0.9}
\Picture{32.3.jpg}{0.9}
\Picture{32.4.jpg}{0.9}
\newpage

\NumTask{Потенциальное силовое поле. Вычисление силовых функций однородного поля силы тяжести и линейной силы упругости.}
\Picture{33.1.jpg}{0.7}
\Picture{33.2.jpg}{0.7}
\Picture{33.3.jpg}{1}
\Picture{33.4.jpg}{1}
\Picture{33.5.jpg}{1}
\newpage

\NumTask{Приближенная теория гироскопа. Теорема Резаля. Правило прецессии.}
\Picture{34.1.jpg}{1}
\Picture{34.2.jpg}{1}
\Picture{34.3.jpg}{1}
\Picture{34.4.jpg}{1}
\Picture{34.5.jpg}{1}
\Picture{34.6.jpg}{1}
\Picture{34.7.jpg}{1}
\newpage
\addtocontents{toc}{\protect\newpage}

\NumTask{Обобщенные силы, способы их вычисления.}
См. билет номер 30.

\NumTask{Основные понятия и допущения приближенной теории гироскопа.}
См. билет номер 34.

\NumTask{Приближенная теория гироскопа. Гироскопический момент, правило Жуковского.}
\Picture{35.1.jpg}{1}
\Picture{35.2.jpg}{1}
\newpage

\NumTask{Движение точки под действием центральной силы. Теорема площадей. }
\Picture{38.1.jpg}{1}
\Picture{38.2.jpg}{1}
\Picture{38.3.jpg}{1}
\Picture{38.4.jpg}{1}
\newpage

\NumTask{Моменты инерции тела относительно осей, проходящих через данную точку в заданном направлении. }
\Picture{39.1.jpg}{0.6}
\Picture{39.2.jpg}{0.6}
\Picture{39.3.jpg}{0.9}
\newpage

\NumTask{Эллипсоид инерции. Главные оси инерции симметричных тел. }
\Picture{40.1.jpg}{0.9}
\Picture{40.2.jpg}{1}
\newpage

\NumTask{Определение реакций подшипников твердого тела, вращающегося вокруг неподвижной оси. }
\Picture{41.1.jpg}{1}
\Picture{41.2.jpg}{1}
\Picture{41.3.jpg}{1}
\Picture{41.4.jpg}{1}
\Picture{41.5.jpg}{1}
\Picture{41.6.jpg}{1}
\Picture{41.7.jpg}{1}
\Picture{41.8.jpg}{0.8}
\Picture{41.9.jpg}{0.8}
\newpage

\NumTask{Понятие статической и динамической уравновешенности твердого тела, вращающегося вокруг неподвижной оси. }
\Picture{42.1.jpg}{0.7}
\Picture{42.2.jpg}{0.7}
\Picture{42.3.jpg}{0.8}
\Picture{42.4.jpg}{0.8}
\newpage

\NumTask{Устойчивость положения равновесия механической системы. Теорема Лагранжа - Дирихле. (без доказательства) }
\Picture{43.1.jpg}{0.7}
\Picture{43.2.jpg}{0.7}
\Picture{43.3.jpg}{1}
\Picture{43.4.jpg}{1}
\newpage

\NumTask{Выражения для кинетической энергии, потенциальной энергии и диссипативной функции Рэлея. }
\Picture{44.1.jpg}{1}
\Picture{44.2.jpg}{1}
\Picture{44.3.jpg}{1}
\Picture{44.4.jpg}{1}
\Picture{44.5.jpg}{1}
\Picture{44.6.jpg}{1}
\Picture{44.7.jpg}{1}
\Picture{44.8.jpg}{1}
\Picture{44.9.jpg}{1}
\Picture{44.10.jpg}{1}
\Picture{44.11.jpg}{1}
\Picture{44.12.jpg}{1}
\newpage

\NumTask{Влияние сил вязкого сопротивления на устойчивость положения равновесия механической системы. }
\Picture{45.1.jpg}{0.9}

\NumTask{Свободные колебания консервативной системы с одной степенью свободы. Элементы гармонических колебаний. }
\Picture{46.1.jpg}{0.9}
\Picture{46.2.jpg}{1}
\newpage

\NumTask{Затухающее колебательное движение. Характеристики затухающих колебаний. }
\Picture{47.1.jpg}{1}
\Picture{47.2.jpg}{1}
\Picture{47.3.jpg}{1}
\Picture{47.4.jpg}{0.9}
\newpage

\NumTask{Затухающее неколебательное движение в случае критического сопротивления и в случае большого сопротивления. }
\Picture{47.5.jpg}{0.9}
\Picture{47.6.jpg}{1}
\Picture{48.1.jpg}{1}
\Picture{48.2.jpg}{1}
\Picture{48.3.jpg}{1}
\Picture{48.4.jpg}{0.8}
\Picture{48.5.jpg}{0.8}
\newpage

\NumTask{Вынужденные колебания в системе с одной степенью свободы. Способы возбуждения колебаний. }
\Picture{49.1.jpg}{0.9}
\Picture{49.2.jpg}{1}
\Picture{49.3.jpg}{1}
\Picture{49.4.jpg}{0.9}
\newpage

\NumTask{Интегрирование дифференциального уравнения вынужденых колебаний в системе с одной степенью сводобы при наличии линейно-вязкого сопротивления. }
\Picture{50.1.jpg}{0.8}
\Picture{50.2.jpg}{1}
\newpage
\addtocontents{toc}{\protect\newpage}

\NumTask{Резонанс в консервативной системе с одной степенью свободы. }
\Picture{51.1.jpg}{1}
\Picture{51.2.jpg}{1}

\NumTask{Вынужденные колебания по лекциям Баркина М.Ю. }
\Picture{52.1.jpg}{1}
\Picture{52.2.jpg}{1}
\Picture{52.3.jpg}{1}
\Picture{52.4.jpg}{1}
\Picture{52.5.jpg}{1}
\Picture{52.6.jpg}{1}

\end{document} 