\documentclass[12pt, a4paper]{article}

\usepackage[utf8]{inputenc}
\usepackage[T1]{fontenc}
\usepackage[russian]{babel}
\usepackage[]{float}

\usepackage[oglav, boldsect, eqwhole, figwhole, %
   remarks, hyperref, hyperprint]{fn2kursstyle}


\frenchspacing
\righthyphenmin=2

%начало
\begin{document}
\tableofcontents 

\NumTask{Аксиомы динамики.}
\Picture{1.pdf}{1}
\newpage

\NumTask{Дифференциальные уравнения движения материальной точки в векторной форме и в проекциях на декартовы и естественные оси координат.}
\Picture{2.1.jpg}{1}
\Picture{2.2.jpg}{1}
\newpage

\NumTask{Две основные задачи динамики точки. Интегралы уравнений движений точки.}
\Picture{3.1.jpg}{1}
\Picture{3.2.jpg}{1}
\Picture{3.3.jpg}{1}
\newpage

\NumTask{Дифференциальные уравнения движения точки в неинерциальной системе отсчета (Динамика относительного движения).}
\Picture{4.1.jpg}{1}
\Picture{4.2.jpg}{1}
\newpage

\NumTask{Принцип относительности Галилея - Ньютона.}
\Picture{5.1.jpg}{1}

\NumTask{Центр масс механической системы. Теорема о движении центра масс.}
\Picture{6.1.jpg}{1}
\Picture{6.2.jpg}{1}
\Picture{6.3.jpg}{1}
\Picture{6.4.jpg}{1}
\newpage

\NumTask{Теорема об изменении количество движения в дифференциальной и интегральной формах.}
\Picture{7.1.jpg}{1}
\Picture{7.2.jpg}{1}
\Picture{7.3.jpg}{1}
\Picture{7.4.jpg}{0.7}
\Picture{7.5.jpg}{0.7}
\newpage

\NumTask{Законы сохранения количества движения механической системы.}
\Picture{8.1.jpg}{0.65}
\Picture{8.2.jpg}{0.7}
\newpage

\NumTask{Законы сохранения кинетического момента (момента количества движения) относительно центра и оси. Примеры.}
\Picture{9.1.jpg}{1}
\Picture{9.2.jpg}{1}
\Picture{9.3.jpg}{1}
\Picture{9.4.jpg}{1}
\newpage

\NumTask{Теоремы об изменении кинетического момента для точки и системы материальных точек.}
\Picture{10.1.jpg}{0.9}
\Picture{10.2.jpg}{1}
\Picture{10.3.jpg}{1}
\newpage

\end{document} 